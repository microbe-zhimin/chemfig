\documentclass[a4paper]{ltjsarticle}
\usepackage{luatexja}
\usepackage[margin=20truemm]{geometry}    % 用紙の余白
\usepackage{setspace}                     % 行間調整
\usepackage{titlesec}                     % セクション設定
\usepackage{graphicx}                     % 画像
\usepackage{color}                        % 画像
\usepackage{pict2e}
\usepackage{float}                        % 画像位置
\usepackage{url}                          % URL
\usepackage[super]{cite}                  % 参考文献引用
\usepackage{longtable}                    % 表
\usepackage{array}                        % 表
\usepackage{booktabs}                     % 実験器具表
\usepackage{tabularx}                     % 表の大きさ変更
\usepackage{caption}                      % キャプション
\usepackage{nccmath}                      % 数式設定
\usepackage[version=3]{mhchem}            % 化学
\usepackage{chemfig}                      % 化学
% --------------------------------------------------------------------------------------------------------------------------------------
\newcommand*{\cnumber}[1]{\raisebox{0.2ex}[0ex][0ex]{\textcircled{\scriptsize{#1}}}} % 丸数字
\newcounter{num} % 数字
\newcommand*{\Rnumber}[1]{\setcounter{num}{#1} \Roman{num}}   % ローマ数字大
\newcommand*{\rnumber}[1]{\setcounter{num}{#1} \roman{num}}   % ローマ数字小
\newcommand*{\RRnumber}[1]{(\setcounter{num}{#1}\Roman{num})} % (ローマ数字大)
\newcommand*{\rrnumber}[1]{(\setcounter{num}{#1}\roman{num})} % (ローマ数字小)
\newcolumntype{C}[1]{>{\hfil}m{#1}<{\hfil}}     % 表セル文字中央
% \captionsetup[figure]{labelfont=normalsize}     % 図キャプションフォント
% \captionsetup[table]{labelfont=normalsize}      % 表キャプションフォント
% --------------------------------------------------------------------------------------------------------------------------------------
\titleformat{\section}{\normalfont\normalsize}{\thesection}{5pt}{}     % セクション環境設定
\titleformat{\subsection}{\normalfont\normalsize}{\thesection}{5pt}{}  % サブセクション環境設定
% --------------------------------------------------------------------------------------------------------------------------------------
\begin{document}
    \setchemfig{scheme debug=true}
    \scriptsize\setchemfig{bond offset=1pt,atom sep=2em,compound sep=4em,arrow coeff=1.25} 
    % ----------------------------------------------------------------------------------------------------------------------------------
    % アセトアニリド
    \schemestart
        \chemfig{*6(-=-(-@{o4}NH2)=-=)}
        \arrow{0}[0,0.1]
        \+
        \arrow{0}[0,0.1]
        \chemfig{-[:-30]@{o1}(=[@{o2}:-90]@{o3}O)-[:30]O-[:-30](=[:-90]O)-[:30]}
        \arrow(.east--.187)
        \chemfig{*6(=-=(-@{o5}\charge{-90:3pt=$\scriptstyle\oplus$}{N}(-[@{o6}:120]H)(-[:60]H)
            (-[:-30]
            (-[:30]O-[:-30](=[:-90]O)(-[:30]))
            (-[:240])
            (-[:300]\charge{270:2pt=$\scriptstyle\ominus$}{O})))-=-)}
        \arrow(.-7--.180)
        \chemfig{*6(=-=(-\chemabove{N}{H}
            (-[:-30]
            (-[@{o7}:30]@{o8}O-[:-30](=[:-90]O)(-[:30]))
            (-[:240])
            (-[@{o9}:300]@{o10}\charge{270:2pt=$\scriptstyle\ominus$}{O})))-=-)}
    \schemestop
    \chemmove[line width=0.2pt,-stealth,dash pattern = on 2pt off 1pt]{
        \draw[shorten <=2pt](o2)..controls +(10:5mm) and +(10:5mm)..(o3.east);
        \draw[shorten >=2pt](o4.45).. controls +(60:10mm) and +(90:10mm).. (o1);
        \draw[shorten <=2pt](o6.90) ..controls +(210:5mm) and +(210:5mm)..(o5);
        \draw[shorten <=2pt](o7)..controls+(120:5mm)and+(120:5mm)..(o8);
        \draw[shorten <=2pt](o10)..controls+(30:5mm)and+(30:5mm)..(o9);
    }
    \linebreak
    \schemestart
        \arrow
        \chemfig{*6(=-=(-\chemabove{N}{H}(-[:-30](-[:30])(=[:270]O)))-=-)}
        \arrow{0}[0,0.1]
        \+
        \arrow{0}[0,0.1]
        \chemfig{[:60]**[-70,30,dash pattern=on 2pt off 2pt]6(O-(-)-O)}
    \schemestop
    \linebreak
    \schemestart
        \chemfig{*6(=-=(-@{o1}\chemabove{N}{H}(-[@{o2}:-30](-[:30])(=[@{o3}:270]@{o4}O)))-=-)}
        \arrow{<->}
        \chemfig{*6(=-=(-\chemabove{\charge{135:2pt=$\scriptstyle\oplus$}{N}}{H}(=[:-30](-[:30])(-[:270]\charge{270:2pt=$\scriptstyle\ominus$}{O})))-=-)}
    \schemestop
    \chemmove[line width=0.2pt,-stealth,dash pattern = on 2pt off 1pt]{
        \draw[shorten <=2pt](o1)..controls +(60:3mm) and +(60:5mm)..(o2);
        \draw[shorten <=2pt](o3).. controls +(10:5mm) and +(10:5mm).. (o4);
    }
    \linebreak
    \linebreak
    \linebreak
    % ----------------------------------------------------------------------------------------------------------------------------------
    % ブロモアニリン
    \newpage
    \schemestart
        \chemfig{@{o5}Br-[@{o6}]@{o7}Br}
        \arrow{0}[0,0.1]
        \+
        \arrow{0}[0,0.1]
        \chemfig{*6(=[@{o8}]-[@{o9}]=[@{o10}](-[@{o11}]@{o12}\chemabove{N}{H}(-[:-30](-[:30])(=[:270]O)))-=-)}
        \arrow
        \chemfig{*6((-[@{o13}:195]H)(-[:225]Br)--=(=[@{o14}]@{o15}\chemabove{\charge{45:2pt=$\scriptstyle\oplus$}{N}}{H}(-[:-30](-[:30])(=[:270]O)))-[@{o16}]=[@{o17}]-[@{o18}])}
        \arrow
        \chemfig{*6((-Br)=-=(-\chemabove{N}{H}(-[:-30](-[:30])(=[:270]O)))-=-)}
        \arrow
        \chemfig{*6((-Br)=-=(-NH_2)-=-)}
    \schemestop
        \chemmove[line width=0.2pt,-stealth,dash pattern = on 2pt off 1pt]{
        \draw[shorten <=2pt](o12)..controls +(120:3mm) and +(120:5mm)..(o11);
        \draw[shorten <=2pt](o10).. controls +(180:3mm) and +(120:3mm).. (o9);
        \draw[shorten <=2pt](o8).. controls +(240:8mm) and +(270:8mm).. (o7);
        \draw[shorten <=2pt](o6).. controls +(90:4mm) and +(90:4mm).. (o5);
        \draw[shorten <=2pt](o13).. controls +(105:2mm) and +(180:2mm).. (o18);
        \draw[shorten <=2pt](o17).. controls +(300:3mm) and +(240:3mm).. (o16);
        \draw[shorten <=2pt](o14)..controls +(120:3mm) and +(120:5mm)..(o15);
    }
    \linebreak
    \linebreak
    \linebreak
    % ----------------------------------------------------------------------------------------------------------------------------------
    % ブロモアニリン
    \newpage
    \schemestart
        \chemfig{*6(=-=[@{o1}](-[@{o2}]@{o3}NH_2)-=-)}
        \arrow{0}[0,0.1]
        \+
        \arrow{0}[0,0.1]
        \chemfig{@{o4}Br-[@{o5}]@{o6}Br}
        \arrow{->}
        \chemfig{*6(=-(-[@{o7}:345]@{o8}H)(-[:315]Br)-[@{o9}](=[@{o10}]@{o11}\charge{45:2pt=$\scriptstyle\oplus$}{N}H_2)-=-)}
        \arrow{0}[0,0.1]
        \+
        \arrow{0}[0,0.1]
        \chemfig{@{o12}\charge{45:2pt=$\scriptstyle\ominus$}{Br}}
        \arrow{->}
        \chemfig{*6(=-(-Br)=(-NH_2)-=-)}
    \schemestop
    \chemmove[line width=0.2pt,-stealth,dash pattern = on 2pt off 1pt]{
        \draw[shorten <=2pt](o3)..controls +(120:3mm) and +(120:5mm)..(o2);
        \draw[shorten <=2pt](o1)..controls +(300:8mm) and +(240:8mm)..(o4);
        \draw[shorten <=2pt](o5).. controls +(90:4mm) and +(90:4mm).. (o6);
        \draw[shorten <=2pt](o12).. controls +(240:5mm) and +(-60:8mm).. (o8);
        \draw[shorten <=2pt](o7).. controls +(75:2mm) and +(0:2mm).. (o9);
        \draw[shorten <=2pt](o10)..controls +(120:3mm) and +(120:5mm)..(o11);
    }
\end{document}
