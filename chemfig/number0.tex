\documentclass[a4paper]{ltjsarticle}
\usepackage{luatexja}
\usepackage[margin=20truemm]{geometry}    % 用紙の余白
\usepackage{setspace}                     % 行間調整
\usepackage{titlesec}                     % セクション設定
\usepackage{graphicx}                     % 画像
\usepackage{color}                        % 画像
\usepackage{pict2e}
\usepackage{float}                        % 画像位置
\usepackage{url}                          % URL
\usepackage[super]{cite}                  % 参考文献引用
\usepackage{longtable}                    % 表
\usepackage{array}                        % 表
\usepackage{booktabs}                     % 実験器具表
\usepackage{tabularx}                     % 表の大きさ変更
\usepackage{caption}                      % キャプション
\usepackage{nccmath}                      % 数式設定
\usepackage[version=3]{mhchem}            % 化学
\usepackage{chemfig}                      % 化学
% -----------------------------------------------------------------------------------------------------------------------------------------------------------------
\newcommand*{\cnumber}[1]{\raisebox{0.2ex}[0ex][0ex]{\textcircled{\scriptsize{#1}}}} % 丸数字
\newcounter{num} % 数字
\newcommand*{\Rnumber}[1]{\setcounter{num}{#1} \Roman{num}}   % ローマ数字大
\newcommand*{\rnumber}[1]{\setcounter{num}{#1} \roman{num}}   % ローマ数字小
\newcommand*{\RRnumber}[1]{(\setcounter{num}{#1}\Roman{num})} % (ローマ数字大)
\newcommand*{\rrnumber}[1]{(\setcounter{num}{#1}\roman{num})} % (ローマ数字小)
\newcolumntype{C}[1]{>{\hfil}m{#1}<{\hfil}}     % 表セル文字中央
% \captionsetup[figure]{labelfont=normalsize}     % 図キャプションフォント
% \captionsetup[table]{labelfont=normalsize}      % 表キャプションフォント
% -----------------------------------------------------------------------------------------------------------------------------------------------------------------
\titleformat{\section}{\normalfont\normalsize}{\thesection}{5pt}{}     % セクション環境設定
\titleformat{\subsection}{\normalfont\normalsize}{\thesection}{5pt}{}  % サブセクション環境設定
% -----------------------------------------------------------------------------------------------------------------------------------------------------------------
\begin{document}
    % \setchemfig{scheme debug=true}
    % CarbonatobisCobalt(III)Chloride
    % \noindent
    \setchemfig{scheme debug=true}
    % 反応機構1
    \setchemfig{bond offset=1pt,atom sep=2.00em,compound sep=4em,arrow coeff=1.25,cram width=2pt, cram dash width=2pt,cram dash sep=2pt,bond join} 
    \schemestart
        \chemfig{*6(---[@{a1}](=O)---)}
        \arrow{0}[0,0.1]
        \+
        \arrow{0}[0,0.1]
        \chemfig{H_2@{b1}N-OH}
        \arrow(.east--.187)
        \chemfig{*6(---(-[:70]@{c1}\charge{90:2pt=$\scriptstyle\ominus$}{O})(-[:-10]@{c2}\charge{270:2pt=$\scriptstyle\oplus$}{N}(-[:0]H)(-[:-45]OH)-[@{c3}:60]@{o4}H)---)}
        \arrow(.-8--.187)
        \chemfig{*6(---(-[@{d1}:70]@{d2}O)(-[@{d3}:-10]@{d4}\chembelow{N}{H}(-[:60]OH))---)}
    \schemestop
    \chemmove[line width=0.2pt,-stealth,dash pattern = on 2pt off 1pt]{
        \draw[shorten <=2pt](b1)..controls+(120:10mm)and+(60:10mm)..(a1);
        \draw[shorten <=2pt](c3)..controls+(120:5mm)and+(120:5mm)..(c2);
        \draw[shorten <=2pt](c1)..controls+(60:5mm)and+(90:8mm)..(o4);
        \draw[shorten <=2pt](d1)..controls+(150:5mm)and+(150:5mm)..(d2);
        \draw[shorten <=2pt](d4)..controls+(90:3mm)and+(80:3mm)..(d3);
    }
    \linebreak
    \linebreak
    \linebreak
    \setchemfig{bond offset=1pt,atom sep=2.00em,compound sep=4em,arrow coeff=1.25,cram width=2pt, cram dash width=2pt,cram dash sep=2pt,bond join} 
    \schemestart
        \arrow(.east--.187){->[\chemfig{@{e1}OH^\ominus}]}
        \chemfig{*6(---(=@{f1}N(-[@{f2}:90]@{f3}H)(-[:-30]OH))---)}
        \arrow(.-8--.187)
        \chemfig{*6(---(=N-[:-30]OH)---)}
    \schemestop
    \chemmove[line width=0.2pt,-stealth,dash pattern = on 2pt off 1pt]{
        \draw[shorten <=2pt](e1)..controls+(90:15mm)and+(120:10mm)..(f3);
        \draw[shorten <=2pt](f2)..controls+(30:5mm)and+(30:5mm)..(f1);
    }
    \linebreak
    \linebreak
    \linebreak
    % 反応機構2
    \setchemfig{bond offset=1pt,atom sep=2.00em,compound sep=4em,arrow coeff=1.25,cram width=2pt, cram dash width=2pt,cram dash sep=2pt,bond join} 
    \schemestart
        \chemfig{*6(---(=[@{a1}]@{a2}N-[@{a3}:-30]@{a4}OH)---)}
        \arrow(.east--.180){->[\chemfig{@{b1}H^\oplus}]}
        \chemfig{*7(---N=(-[1,0.2,,,draw=none]@{c1}\scriptstyle\oplus)---)}
        \arrow(.east--.204){->[\chemfig{@{d1}OH_2}]}
        \chemfig{*7(---N=(-@{e1}\charge{51.4:2pt=$\scriptstyle\oplus$}{O}(-[:111.4]H)(-[@{e2}:-11.4]H))---)}
    \schemestop
    \chemmove[line width=0.2pt,-stealth,dash pattern = on 2pt off 1pt]{
        \draw[shorten <=2pt](a1)..controls+(120:5mm)and+(120:5mm)..(a2);
        \draw[shorten <=2pt](a3)..controls+(240:5mm)and+(240:5mm)..(a4);
        \draw[shorten <=2pt](a4)..controls+(90:10mm)and+(90:10mm)..(b1);
        \draw[shorten <=2pt](d1)..controls+(90:8mm)and+(30:8mm)..(c1);
        \draw[shorten <=2pt](e2)..controls+(270:5mm)and+(270:5mm)..(e1);
    }
    \linebreak
    \linebreak
    \linebreak
    \schemestart
        \arrow(.east--.190)
        \chemfig{*7(---N=[@{f1}](-[@{f2}]O(-[@{f3}:-11.4]@{f4}H))---)}
        \arrow(.-8--.190)
        \chemfig{*7(---NH-(=O)---)}
    \schemestop
    \chemmove[line width=0.2pt,-stealth,dash pattern = on 2pt off 1pt]{
        \draw[shorten <=2pt](f1)..controls+(320:8mm)and+(270:8mm)..(f4);
        \draw[shorten <=2pt](f3)..controls+(270:5mm)and+(320:5mm)..(f2);
    }
    \linebreak
    \linebreak
    \linebreak
    % % ナイロン6
    % \setchemfig{bond offset=1pt,atom sep=2.00em,compound sep=4em,arrow coeff=1.25,cram width=2pt, cram dash width=2pt,cram dash sep=2pt,bond join} 
    % \schemestart
    %     \chemfig{*7(---NH-(=O)---)}
    %     \arrow{0}[0,0.1]
    %     \+
    %     \arrow{0}[0,0.1]
    %     \chemfig{H_2O}
    %     \arrow(.east--.195)
    %     \chemfig{-[@{op,0.2}:30]\chemabove{N}{H}-[@{op2,0.3}:-30]-[@{cl2,0.7}:30](=[:90]O)-[@{cl,0.8}:-30]} 
    %     \polymerdelim[delimiters = {[]}, height = 25pt, depth = 25pt, indice = n ]{op}{cl}
    %     \polymerdelim[delimiters = (), height = 2pt, indice = 5]{op2}{cl2}
    % \schemestop
\end{document}